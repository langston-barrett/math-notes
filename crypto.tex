\documentclass[a5paper]{article}

% New environments. Syntax: \newenvironment{beforecommands}{aftercommands}
% environment for notecards. Portable accross tex files
\newenvironment{card}{\begin{framed}\begin{minipage}[t][3in][t]{5in}\noindent}{\end{minipage}\end{framed}}

% Notes
\newtheorem{noteenv}[theorem]{Note}
\newtheorem*{noteenv*}{Note}
\newenvironment{note}{\begin{noteenv*}}{\end{noteenv*}}

% for the Anki notecard style
\newenvironment{field}{\noindent}{}
\newenvironment{conclusion}{}{}
\newenvironment{premises}{\begin{enumerate}[label=\alph*.]}{\end{enumerate}}
% Theorems, proofs, and other mathematical environments

\usepackage{amsthm}

\theoremstyle{theorem}
\newtheorem{theorem}{Theorem}[subsection]
\newtheorem*{theorem*}{Theorem}
\newtheorem{lemma}[theorem]{Lemma}
\newtheorem*{lemma*}{Lemma}
\newtheorem{corollary}[theorem]{Corollary}
\newtheorem*{corollary*}{Corollary}

\theoremstyle{definition} % these are the non-italicized
\newtheorem{eqenv}[theorem]{Equation}
\newtheorem*{eqenv*}{Equation}
\newtheorem{remark}[theorem]{Remark}
\newtheorem*{remark*}{Remark}
\newtheorem{example}[theorem]{Example}
\newtheorem*{example*}{Example}
\newtheorem{definition}[theorem]{Definition}
\newtheorem*{definition*}{Definition}
%% formatting

\usepackage{fontspec}
% EB Garamond (Initials - just for fun) - body text, old style, serif
% Nimbus Roman - serifed
% URW Gothic - geometric, title text
% Libre Caslon Text - absurdly well spaced
% FreeSerif - plays pretty nicely with math, looks like Times
% Tinos - good alternative to FreeSerif, less TNR looking
\setmainfont[Ligatures=TeX]{Tinos}

\usepackage{datetime2} % use yyyy--mm--dd date with \DTMtoday
\usepackage{geometry}
\geometry{
  lmargin=4cm,
  rmargin=4cm,
  tmargin=3cm,
  bmargin=3cm,
}

\renewcommand{\arraystretch}{1.2} % Make tables a little bigger
% \usepackage{indentfirst}          % indent first \par after sections
% \usepackage{parskip}              % don't indent anything
\usepackage{enumitem}             % enumerate with A), b., c) styles
\usepackage{hyperref}             % linked table of contents
\hypersetup{
    colorlinks,
    citecolor=black,
    filecolor=black,
    linkcolor=black,
    urlcolor=black
}
\usepackage[noabbrev, capitalize]{cleveref}
\crefname{notation}{Notation}{Notations}
\Crefname{notation}{Notation}{Notations}
\crefname{rule}{Rule}{Rule}
\Crefname{rule}{Rule}{Rule}

% graphics
\usepackage{graphicx}
\DeclareGraphicsExtensions{.pdf,.png,.jpg}
\usepackage{float} % image positioning
%% math symbols, notation, etc
\usepackage{amssymb}       % math characters
\usepackage{mathtools}     % improvement on amsmath
% \usepackage{unicode-math}  % use OTF math fonts
\AtBeginDocument{\let\phi\varphi} % varphi with unicode-math

% general stuff
\newcommand{\units}[1]{\;\mathrm{#1}}
\newcommand{\dd}{\,\mathrm{d}}
\newcommand{\ov}{\overline}
\newcommand{\paren}[1]{\left(#1\right)}
\newcommand{\braces}[1]{\left\{#1\right\}}
\newcommand{\brackets}[1]{\left[#1\right]}
\newcommand{\ceiling}[1]{\left\lceil{} #1 \right\rceil}
\newcommand{\floor}[1]{\left\lfloor{} #1 \right\rfloor}
\newcommand{\abs}[1]{\left| #1 \right|}
\newcommand{\e}[1]{\cdot 10^{#1}}
% \newcommand{\emf}{\mathcal{E}}
\newcommand{\contradiction}{\Rightarrow\Leftarrow}
\newcommand{\inv}[1]{\frac{1}{#1}}
\newcommand{\tif}{\text{ if }}
\newcommand{\totherwise}{\text{ otherwise }}
% \newcommand{\mydef}{\textbf{Definition:}}
% complex numbers
\renewcommand{\Im}{\operatorname{Im}}
\renewcommand{\Re}{\operatorname{Re}}
% sets
\newcommand{\zn}{\mathbb{Z}_n}
\newcommand{\zp}{\mathbb{Z}_p}
\newcommand{\N}{\mathbb{N}}
\newcommand{\Z}{\mathbb{Z}}
\newcommand{\Q}{\mathbb{Q}}
\newcommand{\R}{\mathbb{R}}
\newcommand{\C}{\mathbb{C}}
% \C is a command (needs to be renewed) on XeLaTeX

% operators/functions
\DeclareMathOperator{\ord}{ord}
% \DeclareMathOperator{\image}{image}
\DeclareMathOperator{\Int}{Int}
\DeclareMathOperator{\Ext}{Ext}
\DeclareMathOperator{\Lim}{Lim}
\DeclareMathOperator{\Bd}{Bd}
\DeclareMathOperator{\lcm}{lcm}
\DeclareMathOperator{\proj}{proj}
\DeclareMathOperator{\xor}{xor}
\DeclareMathOperator{\spanf}{span}
\DeclareMathOperator{\id}{id}
% linear
\DeclareMathOperator{\col}{Col}
\DeclareMathOperator{\row}{Row}
\DeclareMathOperator{\nul}{Nul}
\DeclareMathOperator{\rank}{rank}

% set theory
\newcommand{\xlongmapsto}[1]{\xmapsto{\,\,\, #1\,\,\,}}
\newcommand{\xlongrightarrow}[1]{\xrightarrow{\,\,\, #1\,\,\,}}
\newcommand{\union}{\mathop{\bigcup}}
\newcommand{\disunion}{\mathop{\dot\bigcup}}
\newcommand{\intersection}{\mathop{\bigcap}}
\newcommand{\powerset}{\mathcal{P}}

% analysis & calculus
\usepackage{esint} % \oiint, etc
\newcommand{\evalat}[2]{{\bigg|_{#1}^{#2}}}
\newcommand{\ext}[1]{{#1}^{ext}}  % extension of function
\DeclareMathOperator{\determinant}{det}
\DeclareMathOperator{\oball}{B}
\DeclareMathOperator{\cball}{\ov{B}}
\DeclareMathOperator{\sphere}{S}
\DeclareMathOperator{\divergence}{div}
\DeclareMathOperator{\curl}{curl}

% vectors
\newcommand{\ihat}{\hat{\imath}}
\newcommand{\jhat}{\hat{\jmath}}
\newcommand{\khat}{\hat{k}}
\newcommand{\lvec}[1]{\overrightarrow{#1}}
\newcommand{\len}[1]{\left\| #1 \right\|}

% topology
\newcommand{\RP}{\ensuremath{\R\hspace{-0.07em}\mathrm{P}}}

% categories
\newcommand{\Cat}{\ensuremath{\mathcal{C}\mathrm{at}}}
\newcommand{\ComGrp}{\ensuremath{\mathcal{C}\mathrm{om}\mathcal{G}\mathrm{rp}}}
\newcommand{\AbGrp}{\ensuremath{\mathcal{A}\mathrm{b}\mathcal{G}\mathrm{rp}}}
\newcommand{\Grp}{\ensuremath{\mathcal{G}\mathrm{rp}}}
\newcommand{\Mod}{\ensuremath{\mathcal{M}\mathrm{od}}}
\newcommand{\Man}{\ensuremath{\mathcal{M}\mathrm{an}}}
\newcommand{\Metric}{\ensuremath{\mathcal{M}\mathrm{etric}}}
\newcommand{\Set}{\ensuremath{\mathcal{S}\mathrm{et}}}
\newcommand{\Top}{\ensuremath{\mathcal{T}\!\mathrm{op}}}
\newcommand{\homoTop}{\ensuremath{\mathrm{homo}\mathcal{T}\mathrm{op}}}
\newcommand{\Vect}{\ensuremath{\mathcal{V}\mathrm{ect}}}
\renewcommand{\phi}{\varphi}

% number theory
\newcommand{\modulus}[1]{\; \left(\mathrm{mod}\; #1\right)}
\newcommand{\crash}[2]{\left(\frac{#1}{#2}\right)}
\newcommand{\QED}{\begin{flushright}QED\end{flushright}}

% algebra
\newcommand{\quotient}[2]{{\raisebox{.2em}{$#1$}\left/\raisebox{-.2em}{$#2$}\right.}}
\newcommand{\GL}{\mathrm{GL}}     % general linear group
\newcommand{\SL}{\mathrm{SL}}     % special linear group
\DeclareMathOperator{\ann}{ann}   % the annihilator of a submodule
\DeclareMathOperator{\Hom}{Hom}   % homomorphisms between modules
\DeclareMathOperator{\Mor}{Mor}   % categorical morphisms
\DeclareMathOperator{\characteristic}{char} % ring characteristic
\DeclareMathOperator{\Tor}{Tor}   % torsion elements
\DeclareMathOperator{\Stab}{Stab} % stabilizer
\DeclareMathOperator{\Orb}{Orb}   % orbit
\DeclareMathOperator{\Aut}{Aut}   % automorphisms
\DeclareMathOperator{\Inn}{Inn}   % inner automorphisms
\DeclareMathOperator{\End}{End}   % endomorphisms
\DeclareMathOperator{\Mat}{Mat}   % matrix ring
\DeclareMathOperator{\Tr}{Trace}  % trace of a matrix
\DeclareMathOperator{\im}{im}     % image

% statistics
\DeclareMathOperator{\Var}{Var}

% nicer empty set
\let\oldemptyset\emptyset{}
\let\emptyset\varnothing{}

\geometry{
  lmargin=1.5cm,
  rmargin=1.5cm,
  tmargin=2cm,
  bmargin=2cm
}
% requires

% one with more spacing. ideally, this would auto-adapt to surroundings
\renewcommand{\:}{\hspace{0.3ex}:\hspace{0.3ex}}

\newcommand{\mvar}[1]{{\color{mvar} #1}}          % TODO: capitalize

\newcommand{\FV}{\ensuremath{\mathsf{FV}}}
\newcommand{\BV}{\ensuremath{\mathsf{BV}}}

% application
\newcommand{\apspace}{\hspace{0.23em}}
\newcommand{\apply}[2]{#1\apspace #2} % apply a function to its argument
\newcommand{\appply}[3]{#1\apspace #2 \apspace #3}
\newcommand{\apppply}[4]{#1\apspace #2 \apspace #3 \apspace #4}
\newcommand{\appppply}[5]{#1\apspace #2 \apspace #3 \apspace #4 \apspace #5}

% LC
\newcommand{\lam}[2]{\uplambda #1.\apspace#2}                % λ-abstraction
\DeclareRobustCommand\longtwoheadrightarrow
     {\relbar\joinrel\twoheadrightarrow}
\newcommand{\betato}{\overset{\upbeta}
  {\;\vphantom{\rule{0pt}{.3ex}}\smash{{\longrightarrow}}}\;}
\newcommand{\betatoo}{\overset{\upbeta}
  {\;\vphantom{\rule{0pt}{.3ex}}\smash{{\longtwoheadrightarrow}}}\;}
\newcommand{\groundtype}{\mathbb{T}}

% Judgments
\newcommand{\judgment}[2]{#2\hspace{0.4em}\text{#1}}
\newcommand{\prop}[1]{\judgment{prop}{#1}}
\newcommand{\type}[1]{\judgment{type}{#1}}
\newcommand{\true}[1]{\judgment{true}{#1}}
\newcommand{\term}[1]{\judgment{term}{#1}}
\newcommand{\valid}[1]{\judgment{valid}{#1}}

% Names of formal systems
\newcommand{\formalsystem}[1]{\textbf{#1}\indeX{#1@\textbf{#1}}}
\newcommand{\HoTT}[0]{\formalsystem{HoTT}}  % homotopy type theory
\newcommand{\LC}[0]{\formalsystem{LC}}      % λ-calculus
\newcommand{\STLC}[0]{\formalsystem{STLC}}  % simply-typed λ-calculus
\newcommand{\TLC}[0]{\formalsystem{TLC}}    % typed λ-calculus
\newcommand{\ITT}[0]{\formalsystem{ITT}}    % intuitionistic type theory
\newcommand{\UTT}[0]{\formalsystem{UTT}}    % univalent type theory
\newcommand{\IPL}[0]{\formalsystem{IPL}}    % intuitionistic propositional logic
\newcommand{\FOL}[0]{\formalsystem{FOL}}    % first-order logic
\newcommand{\ZF}[0]{\formalsystem{ZF}}      % Zermelo–Fraenkel  set theory
\newcommand{\ZFC}[0]{\formalsystem{ZFC}}    % Zermelo-... with choice
\newcommand{\A}{\ensuremath{\mathcal{A}}} % adversary
\newcommand{\MM}{\ensuremath{\mathcal{M}}} % message space
\newcommand{\KK}{\ensuremath{\mathcal{K}}} % key space
\newcommand{\CC}{\ensuremath{\mathcal{C}}} % ciphertext space
\newcommand{\pk}{\ensuremath{\mathrm{pk}}} % public key
\newcommand{\sk}{\ensuremath{\mathrm{sk}}} % secret key

% Without unicode-math:
\makeatletter
\def\operator@font{\sf}
\makeatother

\DeclareMathOperator{\gen}{Gen}
\DeclareMathOperator{\enc}{Enc}
\DeclareMathOperator{\dec}{Dec}
\DeclareMathOperator{\Gen}{Gen}
\DeclareMathOperator{\Enc}{Enc}
\DeclareMathOperator{\Dec}{Dec}
\DeclareMathOperator{\negl}{negl}

% MACs
\DeclareMathOperator{\mac}{MAC}
\DeclareMathOperator{\verify}{Verify}
\DeclareMathOperator{\sign}{Sign}
\DeclareMathOperator{\Mac}{MAC}
\DeclareMathOperator{\Verify}{Verify}
\DeclareMathOperator{\Sign}{Sign}

% Stream ciphers
\DeclareMathOperator{\Init}{Init}
\DeclareMathOperator{\GetBits}{GetBits}


\begin{document}

\title{Cryptography}
\author{Langston Barrett}
\date{Fall 2017}
\maketitle
\tableofcontents
\vspace{1em}
\hrule
\vspace{1em}

\begin{itemize}
  \item Instructor: Adam Groce
  \item Textbook:
    \begin{itemize}
      \item Title: Introduction to Modern Cryptography, 2\textsuperscript{nd} Edition
      \item Author: Jonathan Katz and Yehuda Lindell
      \item ISBN: 978-1-4665-7026-9
    \end{itemize}
\end{itemize}

\section{Probability}
\label{sec:probability}

\begin{lemma}
  If
  \begin{premises}
    \item $A$ and $B$ are random variables,
    \item $B$ is sampled from some finite set of outcomes $\caB$,
  \end{premises}
  then
  \begin{conclusion}
    \begin{equation*}
      \Pr\paren{A}=\sum_{b\in \caB}\Pr\paren{A|B}\cdot\Pr\paren{B=b}
    \end{equation*}
  \end{conclusion}
\end{lemma}

\begin{lemma}[Union Bound]
  For events $E_0,\ldots,E_n$,
  \begin{equation*}
    \Pr\paren{\bigvee_{i=0}^n E_i} \leq \sum_{i=0}^n\Pr E_i
  \end{equation*}
\end{lemma}

\begin{theorem}[Bayes' Theorem]
  \begin{equation*}
    \Pr(A|B) = \frac{\Pr(B|A)\cdot \Pr A}{\Pr B}
  \end{equation*}
\end{theorem}

\section{Concepts}
\label{sec:concepts}

\begin{note}
  \begin{field}
    What's wrong with having a small key space $\caK$?
  \end{field}

  \begin{field}
    It makes your scheme vulnerable to brute-force attacks, especially when the
    distribution on the message space $\caM$ is well-understood (as in all
    adversarial experiments).
  \end{field}
\end{note}

\begin{note}
  \begin{field}
    What are the four kinds of security experiments?
  \end{field}

  \begin{field}
    \begin{enumerate}%[label=\Alph*.]
      \itemsep0em
      \item Ciphertext-only
      \item Known-plaintext
      \item Chosen-plaintext
      \item Chosen-ciphertext
    \end{enumerate}
  \end{field}
\end{note}

\begin{note}
  \begin{field}
    How does a reduction work?
  \end{field}

  \begin{field}
    In it's most general form, reduction is a tool used to show that
    problem/language $A$ is just as ``hard'' as problem/language $B$.
    \begin{enumerate}%[label=\Alph*.]
      \itemsep0em
      \item Assume that problem $B$ ``hard''.
      \item Assume $\caA$ is an algorithm that solves $A$.
      \item Using $\caA$ as a subroutine, construct a solution $\caB$ for $B$.
      \item This contradicts the assumption that $B$ couldn't be solved,
        conclude by contradiction that no such $\caA$ exists.
    \end{enumerate}
  \end{field}
\end{note}

\begin{note}
  \begin{field}
    What is one piece of information that almost every encryption scheme leaks?
    Why might it be a problem? When can and when can't it be solved?
  \end{field}

  \begin{field}
    Plaintext length. It might be a problem if
    $\caM=\braces{\text{``yes''},\text{``no''}}$. It can be solved when the
    maximum length of the encrypted messages is known in advance.
  \end{field}
\end{note}

\begin{note}
  \begin{field}
    Why is it necessary to use randomness in encryption?
  \end{field}

  \begin{field}
    No non-random scheme has indistinguishability for multiple encryptions.
  \end{field}
\end{note}

\section{Symmetric-key cryptography}
\label{sec:symmetric}

\begin{definition*}[Private-key (symmetric) encryption scheme]
  A \define{private-key encryption scheme} consists of:
  \begin{itemize}
    \itemsep0em
    \item a \define{message space} $\caM$,
    \item a \define{key space} $\caK$, and
    \item a trio of algorithms $(\Gen,\Enc,\Dec)$.
  \end{itemize}
  A scheme is \define{correct} if
  \begin{equation*}
    \apply{\Dec_k}{(\apply{\Enc_k}{m})}=m
  \end{equation*}
  for all $m\in\caM$ and $k\in\caK$.
\end{definition*}

\begin{definition*}[Perfect secrecy]
  A private-key encryption scheme $\Pi=(\Gen,\Enc,\Dec)$ is \define{perfectly
  secret} if for all distributions on $\caM$, $m\in\caM$, and $c\in\caC$
  (with $\Pr(C=c)>0$),
  \begin{equation*}
    \Pr(M=m|C=c)=\Pr(M=m)
  \end{equation*}
  Equivalently, for all $m,m'\in\caM$, and $c\in\caC$,
a
  \begin{equation*}
    \Pr(\Enc_K(m)=c)=\Pr(\Enc_K(m')=c)
  \end{equation*}
\end{definition*}

\begin{definition*}[$\mathsf{PrivK}^{\mathsf{eav}}_{\caA,\Pi}$]
  The \define{perfect adversarial indistinguishability experiment
  $\mathsf{PrivK}^{\mathsf{eav}}_{\caA,\Pi}$} is:
  \begin{enumerate}%[label=\Alph*.]
    \itemsep0em
    \item The adversary $\caA$ outputs $m_0,m_1\in\caM$.
    \item
      \begin{enumerate}%[label=\Alph*.]
        \itemsep0em
        \item A key $k\leftarrow\Gen(1^n)$ is generated.
        \item A bit $b\leftarrow \braces{0,1}$ is chosen.
        \item A ciphertext $c\leftarrow\apply{\Enc_k}{m_b}$ is fed to $\caA$.
      \end{enumerate}
    \item $\caA$ outputs $b'\in\braces{0,1}$.
  \end{enumerate}
  The experiment outputs $1$ when $b=b'$.
\end{definition*}

\begin{definition*}[Perfect indistinguishability in the presence of an eavesdropper]
	A scheme $\Pi$ has \define{perfectly indistinguishable encryptions in the
  presence of an eavesdropper} if
  \begin{equation*}
    \Pr\paren{\mathsf{PrivK}^{\mathsf{eav}}_{\caA,\Pi}=1}=\frac 12
  \end{equation*}
  for all $\caA\in\mathsf{PP}$.
\end{definition*}

\begin{definition*}[$\mathsf{PrivK}^{\mathsf{eav}}_{\caA,\Pi}(n)$]
  The \define{adversarial indistinguishability experiment
  $\mathsf{PrivK}^{\mathsf{eav}}_{\caA,\Pi}(n)$} is:
  \begin{enumerate}%[label=\Alph*.]
    \itemsep0em
    \item The adversary $\caA$ is given $1^n$ and outputs $m_0,m_1\in\caM$ with
      $|m_0|=|m_1|$.
    \item
      \begin{enumerate}%[label=\Alph*.]
        \itemsep0em
        \item A key $k\leftarrow\Gen(1^n)$ is generated.
        \item A bit $b\leftarrow \braces{0,1}$ is chosen.
        \item A ciphertext $c\leftarrow\apply{\Enc_k}{m_b}$ is fed to $\caA$.
      \end{enumerate}
    \item $\caA$ outputs $b'\in\braces{0,1}$.
  \end{enumerate}
  The experiment outputs $1$ when $b=b'$.
\end{definition*}

\begin{definition*}[Indistinguishability in the presence of an eavesdropper]
	A scheme $\Pi$ has \define{indistinguishable encryptions in the presence of an
  eavesdropper} if
  \begin{equation*}
    \Pr\paren{\mathsf{PrivK}^{\mathsf{eav}}_{\caA,\Pi}=1}\leq \frac 12 + \apply{\negl}{n}
  \end{equation*}
  for all $\caA\in\mathsf{PP}$ and $n\in\N$, where the probability is taken
  over randomness of $\caA$ and that of the experiment.
\end{definition*}

\begin{definition*}[$\mathsf{PrivK}^{\mathsf{mult}}_{\caA,\Pi}(n)$]
  The \define{multiple-message indistinguishability experiment
  $\mathsf{PrivK}^{\mathsf{mult}}_{\caA,\Pi}(n)$} is:
  \begin{enumerate}%[label=\Alph*.]
    \itemsep0em
    \item The adversary $\caA$ is given $1^n$ and outputs lists
      $(m_{0,0},m_{1,0},\ldots,m_{t,0})$ and $(m_{0,1},m_{1,1},\ldots,m_{t,1})$
      such that $|m_{i,0}|=|m_{i,1}|$ for all $i\in\braces{1,\ldots,t}$.
    \item
      \begin{enumerate}%[label=\Alph*.]
        \itemsep0em
        \item A key $k\leftarrow\Gen(1^n)$ is generated.
        \item A bit $b\leftarrow \braces{0,1}$ is chosen.
        \item The ciphertexts $(\apply{\Enc_k}{m_{0,b}},\ldots,\apply{\Enc_k}{m_{t,b}})$
          are given to $\caA$.
      \end{enumerate}
    \item $\caA$ outputs $b'\in\braces{0,1}$.
  \end{enumerate}
  The experiment outputs $1$ when $b=b'$.
\end{definition*}

\subsection{Peseudorandomness}
\label{subsec:pseudorandomness}

\begin{definition}
	A \textit{determinisitic} algorithm $G\in\mathsf{P}$ is a \define{pseudorandom
  generator} if there exists some real polynomial $l$ such that
  $D:\braces{0,1}^n\to\braces{0,1}^{\apply{l}{n}}$ and the following conditions hold:
  \begin{enumerate}%[label=\Alph*.]
    \itemsep0em
    \item \define{Expansion:} For all $n\in\N$, $\apply{l}{n}>n$.
    \item \define{Pseudorandomness:} For all distinguishers $D\in\mathsf{PP}$,
      there exists a negligible function $\negl$ such that
      \begin{equation*}
        \abs{\Pr(D(r)=1)-\Pr(D(G(s))=1)}\leq \apply{\negl}{n}
      \end{equation*}
      where the first probability is taken over the choice of a uniformly random
      string $r\leftarrow\braces{0,1}^{l(n)}$ and the second over a choice of a
      uniformly random $s\leftarrow\braces{0,1}^n$, and both over randomness of
      $D$. 
  \end{enumerate}
\end{definition}

\begin{note}
  \begin{field}
    What is meant by the phrase ``let $s$ be a random string''?
  \end{field}

  \begin{field}
    Strictly speaking, this phrase doesn't make sense. A given string (or
    function) can't be \textit{random}. What it means is ``let $s$ be a string
    drawn uniformly at random from the set of all strings''.
  \end{field}
\end{note}

\begin{note}
  \begin{field}
    Does the seed of a pseudorandom generator need to be kept secret? Why?
  \end{field}

  \begin{field}
    Yes. Consider the modified one-time pad scheme where a PRG is used to expand
    the key length. If the adversary knows the seed, they know the key.
  \end{field}
\end{note}

\begin{definition}
	A \define{stream cipher} is a pair of deterministic algorithms $(\Init,\GetBits)$
  where 
  \begin{itemize}
    \itemsep0em
    \item $\Init$ takes as input a seed $s$ and an optional initialization
      vector IV, and outputs an initial state $s_0$.
    \item $\GetBits$ takes a state $s_i$ and outputs a bit $b$ and an updated
      state $s_{i+1}$.
  \end{itemize}
\end{definition}

\subsection{CPA-security}
\label{subsec:cpa-security}

\begin{definition*}[$\mathsf{PrivK}^{\mathsf{cpa}}_{\caA,\Pi}(n)$]
  The \define{CPA indistinguishability experiment
  $\mathsf{PrivK}^{\mathsf{cpa}}_{\caA,\Pi}(n)$} is:
  \begin{enumerate}%[label=\Alph*.]
    \itemsep0em
    \item A key $k\leftarrow\Gen(1^n)$ is generated.
    \item The adversary $\caA$ is given $1^n$ and access to the oracle
      $\Enc_k(-)$. The adversary outputs $m_0,m_1\in\caM$ with $|m_0|=|m_1|$.
    \item
      \begin{enumerate}%[label=\Alph*.]
        \itemsep0em
        \item A bit $b\leftarrow \braces{0,1}$ is chosen.
        \item The ciphertext $c\leftarrow\apply{\Enc_k}{m_b}$ is fed to $\caA$.
      \end{enumerate}
    \item $\caA$ continues to have access to $\Enc_k(-)$ and outputs
      $b'\in\braces{0,1}$.
  \end{enumerate}
  The experiment outputs $1$ when $b=b'$.
\end{definition*}

\subsection{Message authentication codes}
\label{subsec:macs}

\begin{definition}[Message authentication code]
	A \define{message authentication code} consists of three
  $\mathsf{PP}$ algorithms $(\Gen,\Mac,\Verify)$ such that:
  \begin{itemize}
    \itemsep0em
    \item $\Gen$ takes input $1^n$ and outputs a key $k$ with $|k|\geq n$,
    \item $\Mac$ takes a key $k\in\caK$ and a message $m\in\caM$ and outputs
      a tag $t$.
    \item $\Verify$ takes a key $k\in\caK$, a message $m\in\caM$, and a tag $t$,
      and outputs a bit $b$ with $b=1$ meaning valid and $b=0$ meaning invalid.
  \end{itemize}
  A MAC is correct if for all $m\in\caM$ and $k\in\caK$,
  \begin{equation*}
    \apply{\Verify_k}{(\apply{\Mac_k}{m})} = 1
  \end{equation*}
\end{definition}

\end{document}