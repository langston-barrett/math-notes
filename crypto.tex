\documentclass[a5paper]{article}

% Theorems, proofs, and other mathematical environments

\usepackage{amsthm}

\theoremstyle{theorem}
\newtheorem{theorem}{Theorem}[subsection]
\newtheorem*{theorem*}{Theorem}
\newtheorem{lemma}[theorem]{Lemma}
\newtheorem*{lemma*}{Lemma}
\newtheorem{corollary}[theorem]{Corollary}
\newtheorem*{corollary*}{Corollary}

\theoremstyle{definition} % these are the non-italicized
\newtheorem{eqenv}[theorem]{Equation}
\newtheorem*{eqenv*}{Equation}
\newtheorem{remark}[theorem]{Remark}
\newtheorem*{remark*}{Remark}
\newtheorem{example}[theorem]{Example}
\newtheorem*{example*}{Example}
\newtheorem{definition}[theorem]{Definition}
\newtheorem*{definition*}{Definition}
% New environments. Syntax: \newenvironment{beforecommands}{aftercommands}
% environment for notecards. Portable accross tex files
\newenvironment{card}{\begin{framed}\begin{minipage}[t][3in][t]{5in}\noindent}{\end{minipage}\end{framed}}

% Notes
\newtheorem{noteenv}[theorem]{Note}
\newtheorem*{noteenv*}{Note}
\newenvironment{note}{\begin{noteenv*}}{\end{noteenv*}}

% for the Anki notecard style
\newenvironment{field}{\noindent}{}
\newenvironment{conclusion}{}{}
\newenvironment{premises}{\begin{enumerate}[label=\alph*.]}{\end{enumerate}}
%% formatting

\usepackage{fontspec}
% EB Garamond (Initials - just for fun) - body text, old style, serif
% Nimbus Roman - serifed
% URW Gothic - geometric, title text
% Libre Caslon Text - absurdly well spaced
% FreeSerif - plays pretty nicely with math, looks like Times
% Tinos - good alternative to FreeSerif, less TNR looking
% \setmainfont[Ligatures=TeX]{Tinos}
\setmainfont{Latin Modern Roman}

% \usepackage{datetime2} % use yyyy--mm--dd date with \DTMtoday
\usepackage{geometry}
\geometry{
  lmargin=4cm,
  rmargin=4cm,
  tmargin=3cm,
  bmargin=3cm,
}

\renewcommand{\arraystretch}{1.2} % Make tables a little bigger
\usepackage{enumitem}             % enumerate with A), b., c) styles
\usepackage{hyperref}             % linked table of contents
\hypersetup{
    colorlinks,
    citecolor=blue,
    filecolor=blue,
    linkcolor=blue,
    urlcolor=blue
}

% graphics
\usepackage{graphicx}
\DeclareGraphicsExtensions{.pdf,.png,.jpg}
\usepackage{float} % image positioning

\input{./tex-preamble/math.tex}

\geometry{
  lmargin=1.5cm,
  rmargin=1.5cm,
  tmargin=2cm,
  bmargin=2cm
}

\newcommand{\define}{\textbf}
\input{./tex-preamble/logic.tex}
\newcommand{\A}{\ensuremath{\mathcal{A}}} % adversary
\newcommand{\MM}{\ensuremath{\mathcal{M}}} % message space
\newcommand{\KK}{\ensuremath{\mathcal{K}}} % key space
\newcommand{\CC}{\ensuremath{\mathcal{C}}} % ciphertext space
\newcommand{\pk}{\ensuremath{\mathrm{pk}}} % public key
\newcommand{\sk}{\ensuremath{\mathrm{sk}}} % secret key

% Without unicode-math:
\makeatletter
\def\operator@font{\sf}
\makeatother

\DeclareMathOperator{\gen}{Gen}
\DeclareMathOperator{\enc}{Enc}
\DeclareMathOperator{\dec}{Dec}
\DeclareMathOperator{\Gen}{Gen}
\DeclareMathOperator{\Enc}{Enc}
\DeclareMathOperator{\Dec}{Dec}
\DeclareMathOperator{\negl}{negl}

% MACs
\DeclareMathOperator{\mac}{MAC}
\DeclareMathOperator{\Mac}{MAC}
\DeclareMathOperator{\verify}{Verify}
\DeclareMathOperator{\Verify}{Verify}
\DeclareMathOperator{\sign}{Sign}
\DeclareMathOperator{\Sign}{Sign}

% Stream ciphers
\DeclareMathOperator{\Init}{Init}
\DeclareMathOperator{\GetBits}{GetBits}


\begin{document}

\title{Cryptography}
\author{Langston Barrett}
\date{Fall 2017}
\maketitle
\tableofcontents
\vspace{1em}
\hrule
\vspace{1em}

\begin{itemize}
  \item Instructor: Adam Groce
  \item Textbook:
    \begin{itemize}
      \item Title: Introduction to Modern Cryptography, 2\textsuperscript{nd} Edition
      \item Author: Jonathan Katz and Yehuda Lindell
      \item ISBN: 978-1-4665-7026-9
    \end{itemize}
\end{itemize}

\begin{note}
  \begin{field}
    What's wrong with having a small key space $\caK$?
  \end{field}

  \begin{field}
    It makes your scheme vulnerable to brute-force attacks, especially when the
    distribution on the message space $\caM$ is well-understood (as in all
    adversarial experiments).
  \end{field}
\end{note}

\begin{note}
  \begin{field}
    What are the four kinds of security experiments?
  \end{field}

  \begin{field}
    \begin{enumerate}%[label=\Alph*.]
      \itemsep0em
      \item Ciphertext-only
      \item Known-plaintext
      \item Chosen-plaintext
      \item Chosen-ciphertext
    \end{enumerate}
  \end{field}
\end{note}

\begin{definition}[Private-key (symmetric) encryption scheme]
  A \define{private-key encryption scheme} consists of:
  \begin{itemize}
    \itemsep0em
    \item a \define{message space} $\caM$,
    \item a \define{key space} $\caK$, and
    \item a trio of algorithms $(\Gen,\Enc,\Dec)$.
  \end{itemize}
  A scheme is \define{correct} if
  \begin{equation*}
    \apply{\Dec_k}{(\apply{\Enc_k}{m})}=m
  \end{equation*}
  for all $m\in\caM$ and $k\in\caK$.
\end{definition}

\end{document}