\documentclass[a5paper]{article}

% Theorems, proofs, and other mathematical environments

\usepackage{amsthm}

\theoremstyle{theorem}
\newtheorem{theorem}{Theorem}[subsection]
\newtheorem*{theorem*}{Theorem}
\newtheorem{lemma}[theorem]{Lemma}
\newtheorem*{lemma*}{Lemma}
\newtheorem{corollary}[theorem]{Corollary}
\newtheorem*{corollary*}{Corollary}

\theoremstyle{definition} % these are the non-italicized
\newtheorem{eqenv}[theorem]{Equation}
\newtheorem*{eqenv*}{Equation}
\newtheorem{remark}[theorem]{Remark}
\newtheorem*{remark*}{Remark}
\newtheorem{example}[theorem]{Example}
\newtheorem*{example*}{Example}
\newtheorem{definition}[theorem]{Definition}
\newtheorem*{definition*}{Definition}
% New environments. Syntax: \newenvironment{beforecommands}{aftercommands}
% environment for notecards. Portable accross tex files
\newenvironment{card}{\begin{framed}\begin{minipage}[t][3in][t]{5in}\noindent}{\end{minipage}\end{framed}}

% Notes
\newtheorem{noteenv}[theorem]{Note}
\newtheorem*{noteenv*}{Note}
\newenvironment{note}{\begin{noteenv*}}{\end{noteenv*}}

% for the Anki notecard style
\newenvironment{field}{\noindent}{}
\newenvironment{conclusion}{}{}
\newenvironment{premises}{\begin{enumerate}[label=\alph*.]}{\end{enumerate}}
%% formatting

\usepackage{fontspec}
% EB Garamond (Initials - just for fun) - body text, old style, serif
% Nimbus Roman - serifed
% URW Gothic - geometric, title text
% Libre Caslon Text - absurdly well spaced
% FreeSerif - plays pretty nicely with math, looks like Times
% Tinos - good alternative to FreeSerif, less TNR looking
% \setmainfont[Ligatures=TeX]{Tinos}
\setmainfont{Latin Modern Roman}

% \usepackage{datetime2} % use yyyy--mm--dd date with \DTMtoday
\usepackage{geometry}
\geometry{
  lmargin=4cm,
  rmargin=4cm,
  tmargin=3cm,
  bmargin=3cm,
}

\renewcommand{\arraystretch}{1.2} % Make tables a little bigger
\usepackage{enumitem}             % enumerate with A), b., c) styles
\usepackage{hyperref}             % linked table of contents
\hypersetup{
    colorlinks,
    citecolor=blue,
    filecolor=blue,
    linkcolor=blue,
    urlcolor=blue
}

% graphics
\usepackage{graphicx}
\DeclareGraphicsExtensions{.pdf,.png,.jpg}
\usepackage{float} % image positioning

\input{./tex-preamble/math.tex}

\geometry{
  lmargin=1.5cm,
  rmargin=1.5cm,
  tmargin=2cm,
  bmargin=2cm
}

\newcommand{\define}{\textbf}

\usepackage{tikz}
\usetikzlibrary{shapes.geometric}
\usetikzlibrary{cd}
% graphs
\usetikzlibrary{positioning}
\usetikzlibrary{graphdrawing,graphs}
\usegdlibrary{trees}  % tree layout
\usegdlibrary{layered}% layered layout
\usegdlibrary{force}  % spring layout
\usepackage{relsize}

\begin{document}

\title{Group Theory}
\author{Langston Barrett}
\date{Spring 2017}
\maketitle
\tableofcontents
\vspace{1em}
\hrule
\vspace{1em}

\begin{itemize}
  \item Instructor: Mckenzie West
  \item Textbook:
    \begin{itemize}
      \item Title: Abstract Algebra, 3\textsuperscript{rd} Edition
      \item Author: David S. Dummit \& Richard M. Foote
      \item ISBN: 0471452343, 9780471452348
    \end{itemize}
\end{itemize}

\begin{definition}[Commutative Diagram]
  A diagram 
  \begin{center}
    \begin{tikzcd}
      A \arrow[r, "f"] \arrow[d, "h"]& B \arrow[d, "g"] \\
      C \arrow[r, "i"] & D
    \end{tikzcd}
  \end{center}
  is said to be commutative if $g\circ f = h\circ i$
\end{definition}

\section{Introduction to Groups}

\subsection{Basic Axioms and Examples}
[Here, I skip some notions from Analysis, such as binary operations,
associativity, commutativty, etc.]

\begin{definition}[Group, Commutative/Abelian Group]
	A group is an ordered pair $(G, \star)$ where $G$ is a set and $\star$ is a
  binary operation on $G$ satisfying
  \begin{enumerate}
    \item Associativity: $\forall a,b,c\in G,(a\star b)\star c = a\star (b\star c)$
    \item Identity: $\exists e\in G,\forall a\in G, e\star a=a\star e=a$
    \item Inverse: $\forall a\in G, \exists a^{-1}\in G,a\star a^{-1}=a^{-1}\star a = e$
  \end{enumerate}

  A group is commutative (abelian) if $\forall a,b\in G, a\star b=b\star a$.
\end{definition}

\begin{example}[Examples of Groups]
	\begin{enumerate}
    \item $\Z,\Q,\R,\C,\mathbb{H}$ are groups under $+$ with $e=0$ and $a^{-1}=-a$
    \item $\Q\setminus\braces{0},\R\setminus\braces{0},\C\setminus\braces{0}$
      are groups under $\cdot$ with $e=1$ and $a^{-1}=1/a$
    \item Roots of unity$\cong$cyclic group of order $n$$\cong$the integers mod
      $n$. The roots of unity are $C_n\coloneqq\braces{x\in\C:x^n=1}$ and the
      operation is multiplication.
  \end{enumerate}
\end{example}

\begin{definition}[Direct Product of Groups]
	If $(A,\star),(B,\diamond)$ are groups, their direct product is
  \begin{equation*}
    A\times B=\braces{(a,b)|a\in A, b\in B}
  \end{equation*}
  with the pointwise group operation
  \begin{equation*}
    (a_1,b_1)(a_2,b_2)=(a_1\star a_2,b_1\diamond b_2)
  \end{equation*}
  and pointwise inversion:
  \begin{equation*}
    (a_1,b_1)^{-1} = (a_1^{-1},b_1^{-1})
  \end{equation*}
\end{definition}

\begin{theorem}[Four Basic Group Properties]
  Let $(G,\star)$ be a group. Then
	\begin{enumerate}%[label=\Alph*.]
    \item The identity of $G$ is unique.
    \item Inverses are unique: $\forall a\in G,\exists!a^{-1}$
    \item Inversion is involutive: $\forall a\in G,(a^{-1})^{-1}=a$
    \item $(a\star b)^{-1}=(b^{-1})\star (a^{-1})$
  \end{enumerate}
\end{theorem}

\begin{proof}
  \begin{enumerate}%[label=\Alph*.]
    % \itemsep0em
    \item 
      Assume that $e_1,e_2\in G$ are identities. Then
      \begin{align*}
        e_1e_2 = e_1 && e_1e_2 = e_2
      \end{align*}
      By the transitivity of $=$, $e_1=e_2$.
    \item
    \item 
      \begin{alignat*}{3}
        (a\star b)\star (-b \star -a) &= a\star (b\star -b) \star -a
        &&\quad\text{Generalized associativity}\\
        &= a\star e \star -a &&\quad\text{Definition of inverses}\\
        &= a\star -a &&\quad\text{Left identity}\\
        &= e &&\quad\text{Definition of inverses}\\
      \end{alignat*}
      So $-(a\star b)=(-b \star -a)$.
    \item
  \end{enumerate}
\end{proof}

\begin{theorem}[Left and Right Cancellation in Groups]
  If $(G,\star)$ is a group, $\forall a,u,v\in G$,
	\begin{align*}
    au=av \implies u=v && ua=va \implies u=v
  \end{align*}
\end{theorem}

\begin{definition}[Order of a Group Element, Order of a Group]
  The order of a group is its cardinality $|G|$. A group is finite if $|G|<\infty$.

	The order of an element $x\in G$ is the smallest positive integer $n$ such
  that $x^n=1$. Equivalently, the order of $x\in G$ is the order of the (cyclic)
  subgroup of $G$ generated by $x$, $|x|=|\langle x \rangle|$.
\end{definition}

\subsection{Dihedral Groups}
\begin{definition}[The Dihedral Group]
	Let $n\geq 3$. The dihedral group $D_{2n}$ is the group of symmetries of a
  regular $n$-gon. It is of order $|D_{2n}|=2n$.

  If we let $r$ be rotation by $2\pi/n$ radians and $s$ be a flip across the
  vertical axis, these suffice in building $D_{2n}$.
\end{definition}

\begin{example}[The Dihedral Group $D_6$]
	The symmetry group for the equalateral triangle is $D_6$.
\end{example}

\begin{remark}
	\begin{itemize}
    % \itemsep0em
    \item $1,r,r^2,\ldots,r^{n-1}$ are all distinct, $|r|=n$.
    \item $1,s$ are distinct and $s^2=1$, so $|s|=2$.
    \item $\forall 0<i,j<n-1,r^i\neq s^j$
    \item $rs=sr^{-1}$
    \item $D_{2n}=\braces{1,r,r^2,\ldots,r^{n-1},s,sr,sr^2,\ldots,sr^{n-1}}$
  \end{itemize}
\end{remark}

\begin{definition}[Generator]
	A subset $S\subseteq G$ is a set of generators of $G$ if every element of $G$
  can be written as a product of elements of $S$ and their inverses. We write
  this $G=\langle S \rangle$.
\end{definition}

\begin{example}[Generators]
  \begin{enumerate}%[label=\Alph*.]
    % \itemsep0em
    \item For $D_{2n}$, $S=\braces{r,s}$ is a set of generators.
    \item For $(\Z,+)$, $S=\braces{1}$ is a set of generators.
    \item For $(\Q\setminus\braces{0},\cdot)$, $S=\Z\setminus\braces{0}$
      generates $\Q$ multiplicatively.
  \end{enumerate}
\end{example}

\begin{definition}[Relation]
	Any equality satisfied by generators of a group (and the identity) is called a
  relation.
\end{definition}

\begin{example}[Relation]
  In $D_{2n}$, $S=\braces{r,s}$.
  \begin{align*}
    rs=sr^{-1} && r^n=1=s^2 
  \end{align*}
  Any other relation on $D_{2n}$ can be derived from these.
\end{example}

\begin{definition}[Presentation]
	If $S$ generates $(G,\star)$ and $R_1,R_2,\ldots,R_m$ are relations satisfied
  by the elements of $S$ and the identity, such that all other relations
  satisfied by elements of $S$ can be constructed (combined using the group
  operation, equalities, etc.) using these, then a presentation of $G$ is
  \begin{equation*}
    G=\langle S|R_1,R_2,\ldots,R_m \rangle
  \end{equation*}

  Note that this set $R_1,R_2,\ldots,R_m$ might not be minimal.
\end{definition}

\begin{example}[Presentations of Common Groups]
	
  \begin{enumerate}
    \item
      \begin{equation*}
        D_{2n}=\langle r,s|rs=sr^{-1},r^n=s^n=1 \rangle
      \end{equation*}
    \item
      \begin{equation*}
        \Z=\langle 1 \rangle
      \end{equation*}
    \item A finite group of order 4:
      \begin{equation*}
        G=\langle x,y | x^2=y^2=(xy)^2=1 \rangle
      \end{equation*}
    \item An infinite group:
      \begin{equation*}
        H=\langle x,y | x^3=y^3=(xy)^3=1 \rangle
      \end{equation*}
  \end{enumerate}
\end{example}

\subsection{Symmetric Groups}
\begin{definition}[Symmetric Group, Permutation]
	If $\Omega$ is a non-empty set, the symmetric group $S_\Omega$ is the group of
  bijections $\phi:\Omega\to\Omega$ where the operation is composition $\circ$.

  If $\Omega=\braces{1,\ldots,n}$ we write $S_n$ for $S_\Omega$. This is called
  the symmetric group of degree $n$.

  An element $\phi\in S_\Omega$ is called a permutation.
\end{definition}

\begin{note}
  \begin{field}
    What is the order of $|S_n|$?
  \end{field}

  \begin{field}
    Well if we fix the image of the first element, the next one has $n-1$
    choices. Then the next one has $n-2$. So we get
    \begin{equation*}
      |S_n|=n!
    \end{equation*}
  \end{field}
\end{note}

\begin{remark}[Cycle Decomposition]
	How can we write the symmetric group concisely? If we have
  \begin{align*}
    1 &\to 4 \\
    2 &\to 3 \\
    3 &\to 2 \\
    4 &\to 1 \\
  \end{align*}
  We write
  \begin{equation*}
    (1 4 2 3)
  \end{equation*}
  But this doesn't work if we have
  \begin{align*}
    1 &\to 2 \\
    2 &\to 1 \\
    3 &\to 4 \\
    4 &\to 3 \\
  \end{align*}
  for which we write
  \begin{equation*}
    (1 2)(3 4)
  \end{equation*}
  Our algorithm is as follows:
  \begin{enumerate}%[label=\Alph*.]
    \item Pick the smallest integer not in a cycle and call it $a$, our new
      cycle is now $(a$
    \item Let $b=\phi(a)$. 
      \begin{enumerate}%[label=\Alph*.]
        \item If $b=a$ then close the cycle as $(a)$, return to (1)
        \item Otherwise, write $b$ next to $a$ in the cycle as $(a b$
      \end{enumerate}
    \item Let $c=\phi(b)$
      \begin{enumerate}%[label=\Alph*.]
        \item If $c=a$, close the cycle
        \item Otherwise, write $(a b c$ and repeat from step 3 with $b=c$.
      \end{enumerate}
    \item Remove anything of the form $(a)$, called 1-cycles.
  \end{enumerate}

  Two cycles are disjoint if they have no integers in common.
\end{remark}

\begin{note}
  \begin{field}
    While $S_n$ is in general non-abelian, disjoint cycles \_.
  \end{field}

  \begin{field}
    Commute.
  \end{field}
\end{note}

\begin{note}
  \begin{field}
    The order of a cycle in $S_n$ is also the \_.
  \end{field}

  \begin{field}
    Least common multiple of the lengths of the cycles in its cycle decomposition.
  \end{field}
\end{note}

\subsection{Matrix Groups}
\begin{definition}[Field, $F^{\times}$]
	A field is a set $F$ together with binary operations $+$ and $\cdot$ such that
  $(F,+)$ is a commutative group with identity 0 and
  $(F\setminus\braces{0},\cdot)$ is also a commutative group with the the left
  distributive law between them:
  \begin{equation*}
    a\cdot(b+c)=(a\cdot b)+(a\cdot c)
  \end{equation*}

  For any field $F$, $F^\times=F\setminus\braces{0}$.
\end{definition}

\begin{example}[Examples of Fields]
	$\Q,\R,\C$ are fields. So is $\Z/p\Z$ where $p$ is prime:
  \begin{itemize}
    \item $\ov{0}$ is the additive identity
    \item $\ov{1}$ is the multiplicative identity
    \item $(\Z/p\Z)^\times=\braces{\ov{1},\ov{2},\ldots,\ov{p-1}}=\Z/p\Z\setminus\braces{\ov{0}}$
  \end{itemize}
  We notate this $\mathbb{F}_p$.
\end{example}

\begin{definition}[$\GL_n(F)$]
	For each an arbitrary field $F$ and $n\in\N$, let the general linear group of
  degree $n$ (denoted $\GL_n(F)$) be the set of $n\times n$ matrices whose entries
  come from $F$ and whose determinant is nonzero.
\end{definition}

\subsection{The Quaternion Group}
\begin{definition}[Quaternion Group ($Q_8$)]
	The quaternion group $Q_8$ is defined to be
  \begin{equation*}
    Q_8\coloneqq \braces{1,-1,i,-i,j,-j,k,-k}
  \end{equation*}
  with product $\cdot$ computed as follows (for all $a\in Q_8$):
  \begin{align*}
    1\cdot a &= a \cdot 1 = a \\
    -1\cdot -1 &= 1 \\
    -1\cdot a &= a\cdot -1=-a \\
    i\cdot i &= j\cdot j=k\cdot k = -1 \\
    i\cdot j = k &&
    j\cdot i = -k \\
    j\cdot k = -i &&
    k\cdot j = -i \\
    k\cdot i = j &&
    i\cdot k = -j 
  \end{align*}
\end{definition}

\begin{note}
  \begin{field}
    What are the generators for the Quaternion Group $Q_8$?
  \end{field}

  \begin{field}
    $\braces{i,j}$ generates $Q_8$:
    \begin{align*}
      i\cdot j &= k \\
      j\cdot i &= k \\
      i\cdot i &= j\cdot j=1 \\
    \end{align*}
  \end{field}
\end{note}

\subsection{Homomorphisms and Isomorphisms}
\begin{definition}[Homomorphism of Groups]
  Let $(G,\star)$ and $(H, \diamond)$ be groups. A map $f:G\to H$ is a
  homomorphism if for all $x,y\in G$,
  \begin{equation*}
     f(x\star y)= f(x)\diamond f(y)
  \end{equation*}
\end{definition}

\begin{definition}[Isomorphism of Groups]
  Let $(G,\star)$ and $(H, \diamond)$ be groups. A map $f:G\to H$ is an
  isomorphism if
  \begin{enumerate}
    \item $f$ is a homomorphism
    \item $f$ is a bijection
  \end{enumerate}
  In this case, $G$ and $H$ are isomorphic, and we write $G\cong H$
\end{definition}

\begin{example}[Isomorphisms]
	\begin{enumerate}
    \itemsep0em
    \item For any group $G$, $G\cong G$ with the identity map (and possibly others)
    \item The exponential function $\operatorname{exp}:\R\to\R^+$ is an
      isomorphism between $(\R,+)$ and $(\R^+,\cdot)$. It is a bijection because
      it has an inverse, and preserves the group operations: $e^{x+y}=e^xe^y$.
    \item All symmetric groups of the same cardinality are isomorphic, and the
      converse holds as well.
    \item Isomorphism is an equivalence relation (with transitivity being
      provided by composition and symmetry by inverses).
  \end{enumerate}
\end{example}

\begin{note}
  \begin{field}
    What conditions need to hold for it to be \textit{possible} that two groups
    are isomorphic?
  \end{field}

  \begin{field}
    For two groups $(G,\star)$ and $(H, \diamond)$, we need to have
    \begin{enumerate}%[label=\Alph*.]
      \item $|G|=|H|$
      \item $G$ is commutative if and only if $H$ is commutative
      \item The order of elements is preserved under the isomorphism
    \end{enumerate}
  \end{field}
\end{note}

\begin{theorem}[Homomorphisms and Presentations]
  If
  \begin{premises}
    \item $(G,\star)$ is a finite group or order $n$ with presentation,
    \item $S=\braces{s_1,\ldots,s_m}$ is its set of generators,
    \item $H$ is another group with $r_1,\ldots,r_m\in H$,
    \item every relation satisfied in $G$ by $s_i$ is satisfied in $H$ by $r_i$,
  \end{premises}
  then
  \begin{conclusion}
    there is a unique homomorphism $f:G\to H$ which maps $s_i$ to $r_i$. If $H$
    is generated by $\braces{r_1,\ldots,r_m}$ and is also of order $n$, then
    $G\cong H$.
  \end{conclusion}
\end{theorem}

\subsection{Group Actions}

\begin{definition}[Left Group Action]
	If $(G,\star)$ is a group and $A$ is a set, then a group action by $G$ on $A$
  is a map $(\cdot):G\times A\to A$ denoted by $(g\cdot a)$ such that
  \begin{enumerate}%[label=\Alph*.]
    % \itemsep0em
    \item For all $g_1,g_2\in G,a\in A$,
    \begin{equation*}
      g_1\cdot(g_2\cdot a)=(g_1\star g_2)\cdot A
    \end{equation*}
    \item For all $a\in A$,
    \begin{equation*}
     1\cdot a=a
    \end{equation*}
  \end{enumerate}
  If $G$ acts on $A$, we call $A$ a $G$-set.
\end{definition}

\begin{example}[Left Group Actions]
	\begin{enumerate}%[label=\Alph*.]
    % \itemsep0em
    \item Scalar mutliplication: the map from $\R\times \R^n\to\R^n$
    given by
    \begin{equation*}
      c\cdot\vec{v}=c\vec{v}
    \end{equation*}
    \item $(D_{2n},\circ)$ and $A=\braces{1,\ldots,n}$: Fix a labeling of the
      vertices of the $n$-gon, then for $\alpha\in D_{2n}$, we define
      $\sigma_\alpha:D_{2n}\times A\to A$ to be the permutation of these
      vertices that's induced by $\alpha$.
    \item $\GL_n(F)$ acts on $F^n$ via applying the (invertible) linear
      transformation that corresponds to the matrix via the standard basis. In
      this way, $\GL_n(F)\hookleftarrow\Aut(F^n)$.
    \item A group $G$ acts on itself via left multiplication:
      \begin{align*}
        G\times G\longrightarrow G && (g,x)\longmapsto gx
      \end{align*}
      This gives the associated map
      \begin{equation*}
        G\hookrightarrow \Aut(G)
      \end{equation*}
      which means that any finite group is isomorphic to a subgroup of $S_{|G|}$.
  \end{enumerate}
\end{example}

\begin{note}
  \begin{field}
    What is the trivial group action of a group $G$ on a set $A$?
  \end{field}

  \begin{field}
    For all $g\in G,a\in A$, define $g\cdot a=a$.
  \end{field}
\end{note}

\begin{theorem}[Group Actions as Permutations]
  If
  \begin{premises}
    \item $(G,\cdot)$ is a group
    \item $A$ is a set
    \item $G$ acts on $A$
  \end{premises}
  then
  \begin{conclusion}
    $\sigma_g:A\to A,\sigma_g(a)=g\cdot a$ is a permutation (bijection) of $A$
    for all $g\in G$.
  \end{conclusion}
\end{theorem}

\begin{proof}
  \begin{align*}
    \sigma_g\circ \sigma_{g^{-1}}(a) &= \sigma_g(g^{-1}\cdot a)
    = g\cdot(g^{-1}\cdot a)
    = (gg^{-1})\cdot a
    = 1\cdot a
    = a
  \end{align*}
  Since we chose $g$ arbitrarily, we can swap $g,g^{-1}$ to show that it is a
  double-sided inverse. Thus, $\sigma_g$ has an inverse, and as so, is bijective.
\end{proof}

\begin{theorem}
  If
  \begin{premises}
    \item $(G,\cdot)$ is a group,
    \item $A$ is a set,
    \item $G$ acts on $A$, and
    \item for each $g\in G$ we define the permutation
      \begin{align*}
        \sigma_g:A\longrightarrow A
        &&\quad\text{by}\quad&&\sigma(a)\coloneqq g\cdot a
      \end{align*}
  \end{premises}
  then
  \begin{conclusion}
    there is a group homomorphism
    \begin{align*}
      \phi:G\longrightarrow S_A &&\quad\text{defined by}\quad&& 
      \phi(g)&\coloneqq \sigma_g
    \end{align*}
  \end{conclusion}
\end{theorem}

\begin{proof}
  Let $g_1,g_2\in G,a\in A$. We want to show that
  \begin{equation*}
    \phi(g_1g_2)=\phi(g_1)\circ \phi(g_2)
  \end{equation*}
  We need:
  \begin{equation*}
    \phi(g_1g_2)(a)=(\phi(g_1)\circ \phi(g_2))(a)
  \end{equation*}
  We have
  \begin{align*}
    \phi(g_1g_2) &= \sigma_{g_1g_2}(a) \\
    &= (g_1g_2)\cdot a \\
    &= g_1\cdot (g_2\cdot a) \\
    &= \sigma_{g_1}(\sigma_{g_2}(a)) \\
    &= (\phi(g_1)\circ \phi(g_2))(a)
  \end{align*}
\end{proof}

\begin{remark}
  We have a correspondence from actions by $G$ on $A$ to homomorphisms from $G$
  to $S_A$. Can we invert this correspondence? Let $\phi:G\to S_A$ be a
  homomorphism. Define a map
  \begin{align*}
    G\times A\longrightarrow A
    &&\text{ by }&& (g,a)&\longmapsto g\cdot a=\phi(g)(a)
  \end{align*}
  Claim: This is an action.

  \begin{proof}
    \begin{alignat*}{3}
      1_G\cdot a &= \phi(1)(a) \\
      &= \operatorname{id}_A(a) &&\quad\text{(Homomorphisms preserve
        identities)} \\
      &= a
    \end{alignat*}
    and
    \begin{alignat*}{3}
      (g_1g_2)\cdot a &= \phi(g_1g_2)(a) \\
      &= (\phi(g_1)\circ\phi(g_2))(a) &&\quad\text{$\phi$ is a homomorphism} \\
      &= \phi(g_1)(\sigma_{g_2}(a)) \\
      &= \sigma_{g_1}(\sigma_{g_2}(a)) \\
    \end{alignat*}
  \end{proof}
  
  Thus, we have a bijection between group actions by $G$ on $A$ and
  homomorphisms from $G\to S_A$.
\end{remark}

\begin{note}
  \begin{field}
    There is a bijection between group actions by a group $G$ on a set $A$ and \_.
  \end{field}

  \begin{field}
    There is a bijection between group actions by a group $G$ on a set $A$ and
    homomorphisms from $G$ into $S_A$, the symmetric group on $A$.
  \end{field}
\end{note}

\begin{definition}[Faithful Group Action]
	A group action of $G$ on $A$ is faithful if every $g\in G$ induces a unique
  permutation on $A$. Equivalently,
  \begin{align*}
    \phi:G\longrightarrow S_A=\Aut(A) && g\longmapsto (x\mapsto g\cdot x)
  \end{align*}
  is injective. Equivalently, the kernel of the action is the identity.
\end{definition}

\begin{definition}[Conjugation]
	For any group $(G,\cdot)$, we can define an action of $G$ on $G$:
  \begin{align*}
    G\times G\longrightarrow G &&
    (g,a)\longmapsto gag^{-1}
  \end{align*}
  This is called conjugation by $G$.
\end{definition}

\section{Subgroups}
\label{sec:subgroups}
\subsection{Definition and Examples} % TODO: revise this section
\label{sub:subgroups-definitions-and-examples}
\begin{definition}[Subgroup, Proper Subgroup]
	If $(G,\star)$ is a group, we say $H\subseteq G$ is a subgroup of $G$ if
  $(H,\star|_H)$ is a group. We denote this $H\leqslant G$. If $H\neq G$, then
  $H$ is a proper subgroup of $G$.
\end{definition}

\begin{lemma}[Necessary and Sufficient Conditions for Subgroups]
	If $(G,\star)$ is a group, then $(H,\star|_H)$ is a group if and only if
  \begin{enumerate}%[label=\Alph*.]
    \item $1_G\in H$
    \item $h_1,h_2\in H\implies h_1\star h_2\in H$
    \item $h\in H\implies h^{-1}\in H$
  \end{enumerate}
\end{lemma}

\begin{example}[Subgroup]
	\begin{enumerate}
    \item If $G=(\Z,+)$, $n\in\Z$, then $n\Z=\braces{nm|m\in Z}$ is a subgroup
      of $G$.
    \item If $G=(D_8,\circ)$, then $\braces{1,r,r^2,r^3}\leqslant G$. 
      You can see the relationships between more subgroups in Figure
      \ref{fig:some-subgroups-of-d8}.
  \end{enumerate}
\end{example}

\begin{figure}[ht]
  \centering
  \begin{tikzpicture}[rounded corners]
    \graph[tree layout, node distance=1.5cm, nodes={draw}]
    {
      "$D_8$" -> "$\braces{1,r,r^2,r^3}$",
      "$D_8$" -> "$\braces{1,s}$",
      "$D_8$" -> "$\braces{1,sr}$",
      "$D_8$" -> "$\braces{1,sr^2}$",
      "$D_8$" -> "$\braces{1,sr^3}$",
      "$\braces{1,r^2}$" ->[bend right] "$\braces{1}$",
      "$\braces{1,s}$" ->[bend right] "$\braces{1}$",
      "$\braces{1,sr}$" ->[bend right] "$\braces{1}$",
      "$\braces{1,sr^2}$" -> "$\braces{1}$",
      "$\braces{1,sr^3}$" -> "$\braces{1}$",
      "$\braces{1,r,r^2,r^3}$" -> "$\braces{1,r^2}$",
    };
  \end{tikzpicture}
  \caption{\label{fig:some-subgroups-of-d8}Some subgroups of $D_8$}
\end{figure}

\subsection{Centralizers and Normalizers, Stabilizers and Kernels}

\begin{definition}[Centralizer of a Group]
	If $(G,\cdot)$ is a group and $a\in G$, then the centralizer of $a$ is
  \begin{equation*}
    C_G(a)\coloneqq \braces{g\in G|gag^{-1}=a}
  \end{equation*}
  If $A\subseteq G$,
  \begin{equation*}
    C_G(A)\coloneqq \braces{g\in G|gag^{-1}=h\;\forall a\in A}
  \end{equation*}
\end{definition}

\begin{note}
  \begin{field}
    What is the meaning of the centralizer of $A\subseteq G$?
  \end{field}

  \begin{field}
    It is the set of elements that commute with $g$.
    \begin{equation*}
      xgx^{-1}=g\iff xg=gx
    \end{equation*}
  \end{field}
\end{note}

\begin{theorem}[Centralizers and Subgroups]
	If $H\subseteq G$ then $C_G(H)$ is a subgroup of $G$.
\end{theorem}
\begin{proof}
  \begin{enumerate}%[label=\Alph*.]
    % \itemsep0em
    \item Identity: $1a1^{-1}=a$ for all $a\in H$, so $1\in C_G(H)$
    \item Closure: Let $x,y\in C_G(H)$. Then for $z\in H$, we have
      \begin{equation*}
        (xy)z(xy)^{-1}=xyzy^{-1}x^{-1}=xzx^{-1}=z
      \end{equation*}
  \end{enumerate}
\end{proof}

\begin{definition}[Center of a Group]
	If $(G,\cdot)$ is a group, then
  \begin{equation*}
    Z(G)=\braces{g\in G|gx=xg\;\forall x\in G}=C_G(G)
  \end{equation*}
  is the \textbf{center} (Zentrum) of $G$, the set of elements that commute with
  everything. 
\end{definition}

\begin{definition}[Normalizer of a Group]
	If $A\subseteq G$, we define
  \begin{equation*}
    gAg^{-1}=\braces{gag^{-1}|a\in A}
  \end{equation*}
  for any $g\in G$. The normalizer of $A$ in $G$ is
  \begin{equation*}
    N_G(A)=\braces{g\in G|gAg^{-1}=A}
  \end{equation*}
\end{definition}

\begin{lemma}[Relationship Between the Normalizer and Centralizer]
  If $x\in C_G(A)$, then $xAx^{-1}=\braces{gag^{-1}|a\in A}=A$, and so
  $C_G(A)\subseteq N_G(A)$.
\end{lemma}

\begin{example}[Centralizers and Normalizers]
	\begin{equation*}
    C_{D_8}(r^2)=\braces{1,r,r^2,r^3,s,\ldots}=D_8
  \end{equation*}
  since $sr^{2}=r^{-2}s=r^{2}s$.
  
  \begin{equation*}
    C_{D_8}(\braces{sr,sr^3})=\braces{1,r^2,sr,sr^3}
  \end{equation*}
\end{example}

\begin{definition}[Kernel of a Group Action]
  The kernel of an action of $(G,\cdot)$ on $A$ is
  \begin{equation*}
    \braces{g\in G|g\cdot a=a,\forall a\in A}
  \end{equation*}
\end{definition}


\begin{definition}[Stabilizer of a Group Action]
	If $(G,\cdot)$ acts on $A$ and $a\in A$, then the stabilizer of $a$ in $G$ is 
  \begin{equation*}
    G_s\coloneqq\braces{g\in G|ga=a}
  \end{equation*}
  This is a subgroup of $G$.
\end{definition}

\subsection{Cyclic Groups and Cyclic Subgroups}

\begin{definition}[Cyclic Groups]
  A group $G$ is cyclic if it is generated by one element, i.e.\
	there is some $x\in G$ such that
  \begin{equation*}
    G=\braces{x^n(=nx):n\in Z}
  \end{equation*}
  We write
  \begin{equation*}
    G=\langle x \rangle
  \end{equation*}
\end{definition}

\begin{lemma}
	Cyclic groups are commutative.
\end{lemma}

\begin{example}[Cyclic Groups]
	\begin{align*}
    \Z = \langle 1 \rangle 
    &&\Z/n\Z = \langle \ov{1} \rangle 
    &&&\langle r \rangle \leqslant D_{2n}
  \end{align*}
\end{example}

\begin{lemma}[The Order of Cyclic Subgroups]
  If
  \begin{premises}
    \item $G$ is a group
    \item $(H,\cdot)$ is a subgroup of $G$
    \item $H$ is cyclic
  \end{premises}
  then
  \begin{conclusion}
    \begin{itemize}
      \item $|H|=\infty$ if and only if $x^a\neq x^b$ for all $a,b\in\Z$ with
        $a\neq b$
      \item $|H|=n$ for $n\in\N_{>0}$ if and only if
        $H=\braces{1,x,\ldots,x^{n-1}}$ and $|x|=n$.
    \end{itemize}
  \end{conclusion}
\end{lemma}

\begin{lemma}
	If $(G,\cdot,1)$ is a group, then
  \begin{enumerate}%[label=\Alph*.]
    % \itemsep0em
    \item If $x^n=1$, $x^m=1$, then $x^{\gcd(m,n)}=1$
    \item If $x^n=1$, then $|x|\big|n$
    \item If $|x|=\infty$, then $|x^a|=\infty$
    \item If $|x|=n<\infty0$, then $|x^a|=\frac{n}{\gcd(n,a)}$
    \item If $|x|=\infty$, then $H=\langle x^a \rangle\iff a=\pm 1$
    \item If $|x|=n<\infty$, then $H=\langle x^a \rangle\iff\gcd(n,a)=1$
  \end{enumerate}
\end{lemma}

\begin{theorem}[Classification of Cyclic Groups]
	Any two cyclic groups of the same order are isomorphic.
  \begin{enumerate}%[label=\Alph*.]
    % \itemsep0em
    \item If $\langle x \rangle$ and $\langle y \rangle$ are finite grouups of
      order $n$, then
      \begin{align*}
        \phi:\langle x \rangle\longrightarrow\langle y \rangle && \phi(x^a)\longmapsto y^a
      \end{align*}
    \item If $\langle x \rangle$ is an infinite group, then
      \begin{align*}
        \psi:\Z\longrightarrow\langle x \rangle && \psi(n)\longmapsto x^n
      \end{align*}
      is an isomorphism.
  \end{enumerate}
\end{theorem}

\begin{theorem}
	Let $H=\langle x \rangle$ be a cyclic group. Then every subgroup of $H$ is
  cyclic, and is generated by $x^a$ where $a$ is the smallest possible integer
  such that $x^a\in K$ (or $K=\braces{1}$).

  Additionally, if $|H|=\infty$, then the subgroups generated by distinct powers
  of $x$ are not equal.

  If $|H|=n<\infty$ then for every $d|n$, there is a unique subgroup of $H$ of
  order $d$: $\langle x^{nd^{-1}} \rangle$.
\end{theorem}

\subsection{Subgroups Generated by Subsets of a Group}
\begin{example}[Subgroups Generated by Subsets of a Group]
	If $A=\braces{x}$, then $\langle A \rangle=\langle x \rangle$
\end{example}

\begin{lemma}[The Intersection of Subgroups]
	If $\braces{H_i:i\in I}$ is a collection of subgroups of a group
  $(G,\cdot,1)$, then $H=\intersection\braces{H_i:i\in I}$ is a subgroup of $G$.
\end{lemma}

\begin{definition}[Subgroup Generated by a Subset]
	If $A\subseteq G$ for some group $(G,\cdot,1)$, the subgroup generated by $A$
  is the intersection of all subgroups of $G$ that contain $A$:
  \begin{equation*}
    \langle A \rangle \coloneqq \intersection_{H\leqslant G,\; A\subseteq H} H
  \end{equation*}
  We write
  \begin{align*}
    \langle A \rangle = \langle a_1,\ldots,a_k \rangle 
  \end{align*}
  if $\braces{a_1,\ldots,a_k}$ and
  \begin{equation*}
    \langle A\cup B \rangle=\langle A,B \rangle
  \end{equation*}
\end{definition}

\begin{theorem}
	Define
  \begin{equation*}
    \ov{A}=\braces{a_1^{\epsilon_1}\cdot\cdots\cdot a_n^{\epsilon_n}|n\in\Z,\epsilon_i\in \Z, a_i\in A\; \forall i}
  \end{equation*}
  The set of all products of finite powers of $a_i$. Then $\ov{A}=\langle A
  \rangle$.
\end{theorem}

\subsection{The Lattice of Subgroups of a Group}
\begin{definition*}[Lattice]
  A lattice is a partially ordered set $(L,\leq)$ where every two-element subset
  of $L$ has both a least upper bound (supremum/join) and a greatest lower bound
  (infimum/meet).

  Naturally, it follows via induction that all finite subsets of $L$ have
  suprema and infima.
\end{definition*}

\begin{definition}[Lattice of Subgroups]
	The lattice of subgroups of a group $G$ is a lattice which has subgroups of
  $G$ as elements and set inclusion as a partial order. The join of two
  subgroups is the subgroup generated by their union, and the meet of two
  subgroups is their intersection.
\end{definition}

\section{Quotient Groups and Homomorphisms}
\subsection{Definitions and Examples}
Consider the map $\phi:\Z\to Z_n$, the cyclic group of order $n$. For any
$x^a\in\Z$, we have $\phi^{-1}(x^a)=a+nm$ for all $m\in\Z$. We also have that
$\phi^{-1}(1)=nm$ and all other fibers are translates of this by elements of
$\Z$.

\begin{definition}[Kernel of a Group Homomorphism]
	The kernel of a group homomorphism $\phi:G\to H$ is the set of elements that
  map to the identity:
  \begin{equation*}
    \operatorname{ker}(\phi)\coloneqq\braces{g\in G|\phi(g)=1} = \phi^{-1}(1)
  \end{equation*}
  This is a subgroup of $G$.
\end{definition}

\begin{definition}[Quotient Group]
	If $G,H$ are groups and $\phi:G\to H$ is a group homomorphism, then we can
  make a group out of the fibers (preimages) of elements of $G$:
  \begin{itemize}
    \item The elements are ``fibers'', or preimages of elements $a$ of $G$ under
      $\phi$, denoted $\phi^{-1}(a)$.
    \item The operation is defined by
      \begin{equation*}
        \phi^{-1}(a)\cdot \phi^{-1}(b)=\phi^{-1}(ab)
      \end{equation*}
  \end{itemize}
  we inherit associativity and identity for free from $G$.

	If $K\coloneqq \operatorname{ker}(\phi)$, we call the above group the quotient
  group $G/K$ (pronounced $G\mod K$).
\end{definition}

\begin{definition}[Left and Right Cosets]
	Let $(H,\cdot,1_H)$ be a subgroup of $(G,\cdot,1_G)$ and $g\in G$. Then a left
  coset of $H$ is
  \begin{equation*}
    gH\coloneqq\braces{gn|n\in H}
  \end{equation*}
  and the right coset of $H$ is
  \begin{equation*}
    Hg\coloneqq\braces{ng|n\in H}
  \end{equation*}
  The set of left cosets of $H$ in $G$ is $G/H$
\end{definition}

\begin{theorem}
	Let $\phi:G\to H$ be a group homomorphism with $K=\operatorname{ker}(\phi)$
  and let $\phi^{-1}(a)\in G/K$ be the fiber above $a$. Then
  \begin{enumerate}%[label=\Alph*.]
    % \itemsep0em
    \item For any $g\in \phi^{-1}(a)$,
      \begin{equation*}
        \phi^{-1}(a)=\braces{gu|u\in K}=gK
      \end{equation*}
    \item For any $g\in \phi^{-1}(a)$,
      \begin{equation*}
        \phi^{-1}(a)=\braces{ug|u\in K}=Kg
      \end{equation*}
  \end{enumerate}
\end{theorem}

\begin{definition}[Representatives of Fibers in Quotient Groups]
  If $\phi:G\to H$ is a group homomorphism and $\phi^{-1}(x)$ is the preimage of
  some element $x\in H$, then an element $g\in \phi^{-1}(x)$ is called a
  representative of $\phi^{-1}(x)$, and we write $gK=\phi^{-1}(x)$. Any element in
  a coset is called a representative of that coset.
\end{definition}

\begin{definition}[Orbit]
  Let $(G,\cdot,1_G)$ be a group and $A$ be a $G$-set. We can define an
  equivalence relation $\sim$ where
  \begin{align*}
    a\sim b\iff a=gb
  \end{align*}
  for some $g\in G$. Then the equivalence class of $a\in A$ is the orbit of $a$
  under the action of $G$.
\end{definition}

\begin{theorem}[Left Cosets and Quotient Groups]
  If
  \begin{premises}
    \item $(G,\star,1_G),(H,\diamond,1_H)$ are groups,
    \item $\phi:G\to H$ is a group homomorphism,
    \item $K=\operatorname{ker}(\phi)$
  \end{premises}
  then
  \begin{conclusion}
    the set $G/K$ with the operation defined by
    \begin{equation*}
      (gK)\bullet (hK)\coloneqq (g\star h)K
    \end{equation*}
    for $g,h\in G$ forms a group.
  \end{conclusion}
\end{theorem}

\begin{example}[Quotient Groups]
  \begin{itemize}
    \item Consider the groups $(\Z,+,0)$ and $Z_n$, the cyclic group of order
      $n$. Then $\ker(\phi)=n\Z$, all the multiples of $n$. So the quotient is
      $\Z/n\Z$.
    \item Consider the quotient of just one group:
      If we have $\phi:G\to H$ where $\phi(g)=1$, then $\ker(\phi)=G$, so
      $G/G\cong \braces{1}$. 
    \item What about the identity morphism $\phi:G\to G,\phi(g)=g$? Then
      $\ker(\phi)=1$, and $G/\braces{1}\cong G$.
    \item How about the map $\phi:\R^2\to\R,\phi(x,y)=x$? Then
      $\ker{\phi}=\braces{(0,y)|y\in\R}=\braces{0}\times\R$ so our quotient
      group is $\R^2/\R\cong \R$.
  \end{itemize}
\end{example}

\begin{lemma}[Cosets of a Subgroup]
	Let $N$ be a subgroup of $G$. Then the set of left cosets of $N$ forms a
  partition of $G$. Furthermore, for all $u,v\in G$, we have
  \begin{equation*}
    uN=vN\iff v^{-1}u\in N
  \end{equation*}
\end{lemma}

\begin{lemma}
	If $(N,\cdot,1_G)$ is a subgroup of $(G,\cdot,1_G)$, then
  \begin{enumerate}%[label=\Alph*.]
    % \itemsep0em
    \item The operation on the set of left cosets given by $uN\star gN=(uv)N$ is
      well-defined if and only if $gng^{-1}\in N$ for all $g\in G,n\in N$.
    \item If this operation is well-defined, then the set of left cosets is a
      group under this operation with identity $1_GN$ and inverses
      $(gN)^{-1}=g^{-1}N$.
  \end{enumerate}
\end{lemma}

\begin{definition}[Conjugate, Normalizing, Normal Subgroup]
  For a group $(G,\cdot,1_G)$ and subgroup $(N,\cdot,1_G)$ and elements
  $g\in G,n\in N$, the element $gng^{-1}$ is called the conjugate of $n$ by $g$.
  The set $gNg^{-1}$ is the conjugate of $N$ by $g$. If $gNg^{-1}=N$, then we
  say that $g$ normalizes $N$. A subgroup $N$ of a group
  $G$ is normal if every element of $G$ normalizes it:
  $\braces{gNg^{-1}|g\in G}=N$, i.e.\ the left cosets of $N$ form a group. We
  write this $N\trianglelefteq G$.
\end{definition}

\begin{lemma}[When is a Subgroup Normal?]
	A subgroup $N$ of a group $G$ is normal if and only if it is the kernal of
  some homomorphism from $G$ to some other group.
\end{lemma}

\begin{definition}[Natural Projection]
  The map
  \begin{align*}
    \pi:G\longrightarrow N
    &&\quad\text{defined by}\quad&&
    \pi(g)\coloneqq gN
  \end{align*}
  is a group homomorphism, called the natural projection. Its kernel is $N$.
\end{definition}

\begin{definition}[Complete Preimage] % TODO: revise
	If $\ov{H}$ is a subgroup of $G/N$, the complete preimage of $\ov{H}$ is
  the preimage of $H$ under the natural projection. It is a subgroup of $G$:
  $\pi^{-1}(\ov{H})\leqslant G$, and contains $N$: $N\leqslant\pi^{-1}(H)$.
\end{definition}

\begin{note}
  \begin{field}
    For a group $G$, what is $G/G$?
  \end{field}

  \begin{field}
    \begin{equation*}
      G/G\cong 1_G
    \end{equation*}
  \end{field}
\end{note}

\begin{note}
  \begin{field}
    For a group $G$, what is $G/1$?
  \end{field}

  \begin{field}
    \begin{equation*}
      G/1\cong G
    \end{equation*}
  \end{field}
\end{note}

\begin{note}
  \begin{field}
    All subgroups of an Abelian group are \_.
  \end{field}

  \begin{field}
    All subgroups of an Abelian group are normal.
  \end{field}
\end{note}

\subsection{More on Cosets and Lagrange's Theorem}
Another intro to cosets:
\begin{definition}
	Let $(G,\cdot,1)$ be a group and $H$ a subgroup. Define a relation on a $G$ by
  $x\sim y$ iff $y^{-1}x\in H$. This is an equivalence.

  The left coset of $H$ containing $x$ is the equivalence class containing $x$
  under $\sim$, denoted $xH$.
\end{definition}

\begin{theorem}[Lagrange's Theorem]
	If $(G,\cdot,1)$ is a finite group and $H$ is a subgroup of $G$, then $|H|$
  divides $|G|$, and the number of cosets of $H$ in $G$ is $|G|/|H|$.
\end{theorem}

\begin{definition}[Index of a Subgroup]
  If $G$ is a finite group and $H$ is a subgroup, then the positive integer
  \begin{equation*}
    \frac{|G|}{|H|}
  \end{equation*}
  guaranteed by Lagrange's Theorem is the index of $H$ in $G$.

  More generally, the index of $H$ in $G$ is the number of left cosets of $H$ in
  $G$.
\end{definition}

\begin{definition}[Multiplication of Subgroups]
	If $H,K$ are subgroups of $G$, then
  \begin{equation*}
    HK\coloneqq\braces{hk|h\in H,k\in K}\subseteq G
  \end{equation*}
  and
  \begin{equation*}
    hK\coloneqq\braces{hk|k\in K}\subseteq G
  \end{equation*}
\end{definition}

\begin{lemma}
	\begin{equation*}
    |HK|=\frac{|H||K|}{|H\cap K|}
  \end{equation*}
\end{lemma}

\begin{lemma}
  If $G$ is a group with subgroups of $K,H$, then $HK$ is a subgroup of $G$ if
  and only if $HK=KH$.
\end{lemma}

\begin{theorem}[Cauchy's Theorem]
  If
  \begin{premises}
    \item $G$ is a finite group
    \item $p\in\N$ is a prime dividing $G$
  \end{premises}
  then
  \begin{conclusion}
    $G$ has an element of order $p$.
  \end{conclusion}
\end{theorem}

\begin{theorem}[Groups of Prime Order]
	If $(G,\cdot,1)$ is a group of prime order, then $G$ is cyclic.

  Corollary: all groups of a given prime order are isomorphic.
\end{theorem}
\begin{proof}
  Let $x\in G,x\neq 1$. Then $|x|=|\langle x \rangle |>1$ and $|x|\big||G|$.
  Since $|G|$ is prime, $|x|=p$ so $G=\langle x \rangle$.
\end{proof}

\begin{lemma}[Subgroup Products and the Normalizer]
	Let $H,K\leqslant G$ with $H\leqslant N_G(K)$. Then $HK\leqslant G$.
\end{lemma}

\subsection{The Isomorphism Theorems}

\begin{theorem}[First Isomorphism Theorem for Groups]
	If $\phi:G\to H$ is a group homomorphism, then $\ker\phi\trianglelefteq G$ and
  $G/\ker\phi\cong\phi(G)$. 
\end{theorem}

\begin{corollary}
	If $\phi:G\to H$ is a group homomorphism, then $\phi$ is injective if and only
  if $\ker\phi=\braces{1}$.
\end{corollary}

\begin{theorem}[Second Isomorphism Theorem]
  If
  \begin{premises}
    \item $A,B,G$ are groups,
    \item $A,B$ are subgroups of $G$,
    \item $A$ is a subgroup of $N_G(B)$,
  \end{premises}
  then
  \begin{conclusion}
    $B\trianglelefteq AB$, $A\cap B\trianglelefteq A$, and
    \begin{equation*}
      \quotient{AB}{B}\cong \quotient{A}{(A\cap B)}
    \end{equation*}
  \end{conclusion}
\end{theorem}

\begin{theorem}[Third Isomorphism Theorem]
  Let $K\trianglelefteq H\trianglelefteq G$ and $K\trianglelefteq H$. Then
  \begin{equation*}
    \quotient{H}{K}\trianglelefteq \quotient{G}{K}
  \end{equation*}
  and
  \begin{equation*}
    \quotient{(G/K)}{(H/K)}\cong G/H
  \end{equation*}
\end{theorem}

\begin{note}
  \begin{field}
    When does a group homomorphism $\Phi:G\to H$ factor through $G/N$? What does
    that even mean?
  \end{field}

  \begin{field}
    If $N\leqslant\ker\Phi$, then we can define a homomorphism
    \begin{align*}
      \phi:G/N\longrightarrow H && \phi(gN)\coloneqq \Phi(g)N.
    \end{align*}
    This homomorphism is well-defined and unique. It is called the induced
    homomorphism. If we let
    \begin{align*}
      \pi:G\longrightarrow G/N && \pi(g)\coloneqq gN
    \end{align*}
    be the natural projection, than for any $\Phi$, the following diagram commutes:
    \begin{center}
      \begin{tikzcd}
        G \arrow[r, "\pi"]
          \arrow[dr, "\Phi"]
          & G/N \arrow[d, dashed, "\phi"] \\
          & H
      \end{tikzcd}
    \end{center}
  \end{field}
\end{note}

\subsection{The H\"older Program and Simple and Solvable Groups}
\begin{lemma}[Finite Groups and Elements of Prime Order]
	Let $(G,\cdot,1_G)$ be a finite commutative group and $p$ a prime dividing
  $|G|$. Then $\exists g\in G$ such that $|g|=p$.
\end{lemma}

\begin{definition}[Simple Group]
	A group $(G,\cdot,1)$ is simple if the only normal subgroups of $G$ are the
  trivial ones ($\braces{1},G$).
\end{definition}

\begin{theorem}[Feit-Thompson]
  If $G$ is an odd-order simple group, then $G\cong Z_p$ for some prime $p$.
  This result was \sim 250 pages.
\end{theorem}

\begin{definition}[Solvable Group]
	A group $G$ is solvable if there is a chain of subgroups
  $\braces{1}=G_0\trianglelefteq G_1\trianglelefteq\cdots\trianglelefteq G_s=G$
  such that $G_{i+1}/G_i$ is commutative for $i=0,\ldots,s-1$.
\end{definition}

\begin{theorem}
	If
  \begin{premises}
    \item $G$ is a group with normal subgroup $N$,
    \item $N$ is solvable, and
    \item $G/N$ is solvable,
  \end{premises}
  then
  \begin{conclusion}
    $G$ is solvable.
  \end{conclusion}
\end{theorem}

\subsection{Transpositions and the Alternating Group}
\begin{definition}[Two-Cycle/Transposition]
	A two-cycle in the symmetric group $S_n$ is also called a transposition.
  We can write a general cycle $(a_1\;\ldots\;a_m)\in S_n$ as
  \begin{equation*}
    (a_1\;\ldots\;a_m)=(a_1\;a_m)(a_1\;a_{m-1})\cdots(a_1a_2)
  \end{equation*}
  i.e.\ the product of two-cycles. Thus, the symmetric group is generated by
  transpositions. 
\end{definition}

\begin{definition}[The Alternating Group]
	The alternating group is the subgroup of $S_n$ containing all permutations
  that can be written as the product of an even number of transpositions.
\end{definition}

\begin{example}[The Alternating Group]
  The alternating group is the subgroup of $S_n$ that is made of permutations
  that are the product of an even number of transpositions.
\end{example}

\begin{theorem}[The Order of $|S_n/A_n|$]
	For all $n\geq 2$, $|S_n/A_n|=2$.
\end{theorem}

\section{Group Actions}
\subsection{Group Actions and Permutation Representations}
\begin{theorem}
	Let $G$ be a finite group and $p$ the smallest prime dividing $|G|$. Then
  any subgroup $H\subseteq G$ of index $p$ is normal.
\end{theorem}

\begin{note}
  \begin{field}
    The kernel of a group action $\cdot:G\times A\to A$ is the same as \_.
  \end{field}

  \begin{field}
    The kernel of the associated permutation representation
    \begin{align*}
      \sigma_g:A\longrightarrow A && \sigma_g(a)\coloneqq g\cdot a,
    \end{align*}
    or the intersection of the stabilizers of all the $a\in A$.
  \end{field}
\end{note}

\begin{example}
	Consider $S_n$ where $n\geq 3$. We have $A_n\leqslant S_n$ with $|S_n:A_n|=2$.
  By the above theorem, $A_n\trianglelefteq S_n$, so $S_n$ is not simple for
  $n\geq 3$.
\end{example}

\begin{theorem}[Orbit-Stabilizer Coset Correspondence]
	Let $G$ act on $A$. Then the relation $a\sim b\iff \exists g\in G$ such that
  $a=gb$ is an equivalence relation on $A$. Let $G_a=\braces{g\in G|ga=a}$ the
  stabilizer of $a$ in $G$, and $G\cdot a=\braces{g\cdot a|g\in G}$ the orbit of
  $a$ in $G$. Then $|G\cdot a|=|G:G_a|$.
\end{theorem}

\begin{definition}[Transitive Group Action]
	A group action is transitive if it has only one orbit.
\end{definition}

\subsection{Groups Acting on Themselves by Left Multiplication—Cayley's Theorem}
\begin{theorem}[Cayley's Theorem]
	Every group $G$ is isomorphic to some subgroup of a symmetric group.
  Specifically, if $|G|=n$, then $G$ is isomorphic to a subgroup of $S_n$.
\end{theorem}

\subsection{Groups Acting on Themselves by Conjugation—The Class Equation}
\begin{definition}[The conjugation action, conjugacy classes]
  For $a,b\in G$, we say that $a$ is conjugate to $b$ if $\exists g\in G$ such
  that $a=gbg^{-1}$. In fact, $G$ acts on itself via conjugation:
  \begin{align*}
    \cdot:G\times G\longrightarrow G
    &&(g,a)\xmapsto{\;\;\cdot\;\;} gag^{-1}
  \end{align*}
  The orbits of this action are called conjugation classes, often denoted
  $[a]=\braces{gag^{-1}:g\in G}$.

  Two sets are conjugate if $\exists g\in G$ such that $S=gTg^{-1}$.
\end{definition}

\begin{note}
  \begin{field}
    What is the conjugation class of $c\in C_G(G)=Z(G)$?
  \end{field}

  \begin{field}
    $\braces{c}$, since it commutes with everything, $gcg^{-1}=c,\;\forall g$.
  \end{field}
\end{note}

\begin{note}
  \begin{field}
    A normal subgroup is conjugate to \_.
  \end{field}

  \begin{field}
    itself.
  \end{field}
\end{note}

\begin{theorem}
	The number of conjugates of $S\subseteq G$ is $|G:N_G(S)|$. In particular, the
  number of conjugates of $s\in G$ is $|G:C_G(s)|$.
\end{theorem}

\begin{theorem}[The Class Equation]
	Let $G$ be a finite group and $g_1,\ldots,g_r\in G$ be representatives of the
  conjugacy classes of $G$ not contained in $C_G(G)$. Then
  \begin{equation*}
    |G|=|C_G(G)|+\sum_{i=1}^r|G:C_G(g_i)|
  \end{equation*}
\end{theorem}

\begin{theorem}
	If $G$ is a group with order $p^a$ for some prime $p$, then $C_G(G)$ is
  non-trivial. 
\end{theorem}
\begin{proof}
  Since $|G|$ is $p^a$, if $g_i\notin C_G(G)$, then $|G:C_G(G)|=p^b$ for some
  $b<a$. Then the class equation gives %$|C_G(G)|$
  \begin{equation*}
    p^a=|C_G(G)|+pn
  \end{equation*}
  for some $0<n<a$, so $p\big||C_G(G)|$.
\end{proof}

\begin{corollary}
  If $|G|=p^2$ for some prime $p$, then $G$ is commutative. Moreover, $G$ is
  isomorphic to $Z_{p^2}$ or $Z_p\times Z_p$.
\end{corollary}

\subsubsection{Conjugacy in $S_n$}
\begin{lemma}
	If $\sigma,\tau\in S_n$ and 
  \begin{equation*}
    (a_1\,\,a_2\,\,a_3\,\,\ldots)(b_1\,\,b_2\,\,b_3\,\,\ldots)
  \end{equation*}
  then 
  \begin{equation*}
    \tau\circ\sigma\circ\tau^{-1}
    =(\tau a_1\,\,\tau a_2\,\,\tau a_3\,\,\ldots)(\tau b_1\,\,\tau b_2\,\,\tau b_3\,\,\ldots)
  \end{equation*}
\end{lemma}

\begin{definition*}[Cycle Types]
  Let $\sigma\in S_n$ and assume $\sigma$ can be written as disjoint cycles of
  lengths $n_1\leq n_2\leq\cdots\leq n_k$. Then $n_1,\ldots,n_k$ is the cycle
  type of $\sigma$.
\end{definition*}

\subsection{Automorphisms}

\begin{definition}[Automorphism, automorphism group]
	An isomorphism of a group onto itself is an automorphism. The set of
  automorphisms of a group $G$ is itself a group under composition, denoted
  $\Aut(G)$.
\end{definition}

\begin{theorem}[Conjugating a normal subgroup]
  If $H$ is a normal subgroup of $G$, then $G$ acts on $H$ by conjugation:
  \begin{align*}
    G\times H\longrightarrow H && (g,h)\longmapsto ghg^{-1}
  \end{align*}
  and for each $g\in G$, conjugation by $g$ is an automorphism of $H$. The
  permutation representation of this action is a homomorphism of $G$ into
  $\Aut(H)$.
\end{theorem}

\begin{definition}[Inner automorphism]
	If $G$ is a group, then conjugation of $G$ by $g$ is an inner automorphism.
  The subgroup of $\Aut(G)$ consisting of all inner automorphisms is denoted
  $\Inn(G)$.
\end{definition}

\begin{definition}[Characteristic Subgroup]
  A subgroup $H$ of a group $G$ is characteristic if every automorphism of $G$
  maps $H$ to itself, i.e.\ $\sigma(H)=H$ for all $\sigma\in\Aut(G)$.
\end{definition}

\begin{theorem}[Properties of characteristic subgroups]
  \begin{enumerate}%[label=\Alph*.]
    % \itemsep0em
    \item Characteristic subgroups are normal.
    \item If $H$ is the unique subgroup of $G$ of a given order, then $H$ is
      characteristic in $G$.
    \item If $K$ is characteristic in $H$ and $H\trianglelefteq G$, then
      $K\trianglelefteq G$.
  \end{enumerate}
\end{theorem}

\subsection{The Sylow Theorems}
\begin{definition}[$p$-group, $p$-subgroup, Sylow $p$-subgroup]
	A group $G$ of prime order $p$ is a $p$-group, a subgroup of $G$ of prime
  order $p$ is a $p$-subgroup, and if $G$ is of order $p^am$ where $p$ is prime
  and $p\nmid m$, then a subgroup of order $p^a$ is a Sylow $p$-subgroup of $G$.
\end{definition}

\begin{theorem}[Sylow's theorem (simplified)]
  If 
  \begin{premises}
    \item $(G,\cdot,1)$ is a group,
    \item $|G|=p^am$ for prime $p$ with $p\nmid m$,
  \end{premises}
  then
  \begin{conclusion}
    there are Sylow $p$-subgroups of $G$, they are all conjugate to one another,
    and the number of Sylow $p$-subgroups is of the form $1+kp$ for $k\in\N$.
  \end{conclusion}
\end{theorem}

\begin{corollary}
	If $P$ is a Sylow $p$-subgroup of $G$, then the following are equivalent:
  \begin{enumerate}%[label=\Alph*.]
    % \itemsep0em
    \item $P$ is the unique $p$-subgroup of $G$,
    \item $P$ is normal in $G$, and
    \item $P$ is characteristic in $G$.
  \end{enumerate}
\end{corollary}

% \subsection{The Simplicity of $A_n$}
% \section{Direct and Semidirect Products and Abelian Groups}
% \subsection{Direct Products}
% \subsection{The Fundamental Theorem of Finitely Generated Abelian Groups}
% \subsection{Table of Groups of Small Order}
% \subsection{Recognizing Direct Products}
% \subsection{Semidirect Products}
% \section{Further Topics in Group Theory}
% \subsection{$p$-groups, Nilpotent Groups, and Solvable Groups}
% \subsection{Applications in Groups of Medium Order}
% \subsection{A Word on Free Groups}

\end{document}
